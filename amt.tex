\subsection{Authenticated Multipoint Evaluation Trees (AMTs)}
\label{s:amt}
In this section, we improve KZG's $\Theta(nt)$ time for computing $n$ proofs for a degree-bound $t$ polynomial to $\Theta(n\log{t})$ time.
Our key technique is to commit to the quotients in a polynomial multipoint evaluation (see~\cref{s:prelim:multipoint-eval}), obtaining an \textit{authenticated multipoint evaluation tree} (AMT).
However, our new AMT evaluation proofs are logarithmic-sized, whereas KZG proofs are constant-sized.
As a result, when we apply AMTs to scale VSS and DKG later in \cref{s:scalable-vss,s:scalable-dkg}, we slightly increase communication complexity and reconstruction time.
Nonetheless, in \cref{s:eval}, we demonstrate that the time saved in proof computation more than makes up for these smaller increases.
Throughout this section, we restrict ourselves to computing AMTs at points $\{1,2,\dots,n\}$ on polynomials of degree $t-1 < n$, since this is the VSS/DKG setting,
(In \cref{s:amt:arbitrary-points}, we discuss generalizing to any set of points.)
Finally, in \cref{s:proofs}, we show AMT evaluation proofs are secure under $q$-SBDH.
In contrast, KZG proofs are secure under a weaker assumption called $q$-SDH~\cite{BonehBoyen2008}.

\subsubsection{Computing AMT proofs}
\label{s:amt:computing-proofs}
KZG evaluation proofs leverage the \textit{polynomial remainder theorem}: $\forall i\in \Fp, \exists q_i$ of degree $t-1$ such that $\phi(x) = q_i(x)(x-i) + \phi(i)$.
Specifically, a constant-sized KZG proof for $\phi(i)$ is just a commitment to the quotient polynomial $q_i$ (see \cref{s:prelim:polycommit}) and takes $\Theta(t)$ time to compute.
Thus, computing KZG proofs for each $i\in[n]$ takes $\Theta(nt)$ time.
We improve on this by looking at $\phi(x)$ from the lens of a polynomial multipoint evaluation~\cite{moderncomputeralgebra-ch10}.

For example, consider the multipoint evaluation of $\phi$ at all $i\in[8]$ from \cref{f:multipoint-eval}.
Note that every node in the multipoint evaluation tree stores a quotient and a remainder obtained by dividing the parent node's remainder by its accumulator polynomial (see \cref{s:prelim:multipoint-eval}).
The first key idea is that, for any evaluation point $i\in[8]$, $\phi(x)$ can be expressed as $\phi(i)$ plus a linear combination of quotients and accumulator polynomials along the path to $\phi(i)$'s leaf in the multipoint evaluation tree.
For example, consider $i=1$, which has the left-most path in tree.
Start with the root node in \cref{f:multipoint-eval}, which says:
$$\phi(x)=q_{1,8}(x) (x-1)\dots(x-8) + r_{1,8}(x)$$
Then, expand $r_{1,8}(x)$ by going left in the tree (down towards $\phi(1)$'s leaf), obtaining:
\begin{align*}
\phi(x)& = q_{1,8}(x) (x-1)(x-2)(x-3)(x-4)\cdots(x-8) \\
       & + q_{1,4}(x) (x-1)(x-2)(x-3)(x-4) + r_{1,4}
\end{align*}
Repeat this process recursively by replacing $r_{1,4}(x)$ and then $r_{1,2}(x)$ to get:
\begin{align*}
\phi(x)& = q_{1,8}(x) (x-1)(x-2)(x-3)(x-4)\cdots(x-8) \\
       & + q_{1,4}(x) (x-1)(x-2)(x-3)(x-4) \\
       & + q_{1,2}(x) (x-1)(x-2) \\
       & + q_{1,1}(x)(x-1) + \phi(1).
\end{align*}
Note that $\phi(x)$ can be re-expressed similarly for any other points $i\in [2,n]$.
Importantly, note that there are only $\Theta(n)$ quotient and accumulator polynomials shared by all such expressions of $\phi(i)$.

Our second key idea follows naturally: we commit to all these $\Theta(n)$ quotient polynomials in the multipoint evaluation of $\phi$.
This gives us logarithmic-sized evaluation proofs for any point $i\in[n]$.
We call these proofs \textit{AMT proofs}.
For example, in \cref{f:multipoint-eval}, the AMT proof for $\phi(4)$ would be $\{g^{q_{1,8}(\tau)}, g^{q_{1,4}(\tau)}, g^{q_{3,4}(\tau)}, g^{q_{4,4}(\tau)}\}$, where $\tau$ denotes the trapdoor used in KZG commitments (see \cref{s:prelim:polycommit}).

\subsubsection{Verifying AMT proofs}
\label{s:amt:verifying-proofs}
The next question is how to verify our new logarithmic-sized AMT proofs.
Recall that, given any point $i$, $\phi(x)$ can be expressed as:
\begin{align}
\label{eq:amt-path}
\phi(x) &= \phi(i) + \sum_{w\in \mathsf{\treepath(i)}} q_w(x) a_w(x)
\end{align}
where $\treepath(i)$ is the set of nodes along the path from the root to $\phi(i)$ and $q_w$ and $a_w$ denote the quotient and accumulator polynomials stored at node $w$ in the multipoint evaluation tree (see \cref{f:multipoint-eval}).
How can we verify a proof $\pi_i = \left(g^{q_w(\tau)}\right)_{w\in\treepath(i)}$ for $\phi(i) = y_i$?
We simply use a bilinear map to check that \cref{eq:amt-path} holds at $x=\tau$:
\begin{align}
\label{eq:amt-verify}
e(g^{\phi(\tau)}, g) &\stackrel{?}{=} e(g^{y_i},g) \prod_{w\in\treepath(i)} e(g^{q_w(\tau)}, g^{a_w(\tau)}) \Leftrightarrow\\ \nonumber
e(g, g)^{\phi(\tau)} &\stackrel{?}{=} e(g,g)^{y_i} \prod_{w\in\treepath(i)} e(g,g)^{q_w(\tau) a_w(\tau)}\Leftrightarrow\\ \nonumber
e(g, g)^{\phi(\tau)} &\stackrel{?}{=} e(g,g)^{{y_i}+\sum_{i\in\treepath(w)} {q_w(\tau) a_w(\tau)}}\Leftrightarrow\\ \nonumber
         \phi(\tau)  &\stackrel{?}{=} {y_i}+\sum_{w\in\treepath(i)} {q_w(\tau) a_w(\tau)} 
\end{align}
This is reminiscent of how KZG proofs are verified by checking that $\phi(x) = q_i(x)(x-i) + \phi(i)$ holds at $x=\tau$ (see \cref{s:prelim:polycommit}).
However, note that \textit{the verifier needs to have the} $g^{a_w(\tau)}$ \textit{accumulator commitments}, which are not part of the AMT proof.
This implies AMT verifiers must have $\Theta(n)$ public parameters, whereas KZG verifiers only need $\{g^\tau\}$ as their public parameters (see \cref{s:prelim:polycommit}).
Fortunately, in \cref{s:amt:public-parameters} we reduce the verifers' public parameters to just $\Theta(\log{t})$.

\subsubsection{Better AMTs using roots of unity}
\label{s:amt:roots-of-unity}
Instead of evaluating $\phi$ at points $\{1,2,3,\dots,n\}$, we assume $n=2^m$ and evaluate $\phi$ at all $n$ $n$th roots of unity in $\Fp$.
Specifically, we compute $\phi(\omega_n^{i-1})$ rather than $\phi(i)$, where $\omega_n$ is a primitive $n$th root of unity.
(We can generalize to any $n$ by using the first $n$ $N$th roots of unity, where $N=2^m$ is the smallest value such that $N\ge n$.)
The main benefit of using roots of unity is they give rise to simpler accumulator polynomials of the form $(x^{2^k} + c)$ in the multipoint evaluation tree (for some $c$).
This speeds up the multipoint evaluation (see \cref{s:amt:proof-time-and-sizes}), since dividing degree-bound $2n$ polynomials by $(x^n + c)$ can be done in $\Theta(n)$ rather than $\Theta(n\log{n})$ time.
In \cref{s:amt:proof-time-and-sizes}, we show this new, optimized AMT proof is $\amtProofSize{t}$ group elements and computing an AMT takes $\amtDealTime$ time.

The $(x^{2^k} + c)$ form of the accumulators is best illustrated with an example.
Let $n=8$ and $\omega_8$ denote a primitive 8th root of unity.
Previously, in \cref{f:multipoint-eval}, the evaluation points $\{1,2,\dots,8\}$ were ordered as $\langle (x-1), (x-2),\dots, (x-8)\rangle$ monomials in the leaves.
Then, the accumulators were computed by multiplying ``up the tree,'' culminating in the root accumulator $\prod_{i\in[8]}(x-i)$.
In our case, the evaluation points are $\{\omega_8^{i-1}\}_{i\in[8]}$ but we reorder them using a bit-reversal permutation~\cite{bitreversal-wiki} as $\langle(x-\omega_8^0), (x-\omega_8^4), (x-\omega_8^2), (x-\omega_8^6), (x-\omega_8^1), (x-\omega_8^5), (x-\omega_8^3), (x-\omega_8^7)\rangle$.
This ordering ensures that, as we multiply ``up the tree,'' all accumulators are of the form $(x^{2^k} + \omega_8^j)$ for some $j$.

Let us see exactly how this happens.
The parent accumulator of the first two leaves $(x-\omega_8^0)$ and $(x-\omega_8^4)$ is their product $(x-\omega_8^0)(x-\omega_8^4) = x^2 - \omega_8^4 x - \omega_8^0 x + \omega_8^0 \omega_8^4$.
Since $\omega_n^i \omega_n^j = \omega_n^{(i+j) \bmod n}$~\cite{CLRS09}, this equals $x^2 - x(\omega_8^4 + \omega_8^0) + \omega_8^4$.
Since $\omega_n^{k+n/2} = -\omega_n^k$~\cite{CLRS09}, this equals $(x^2 + \omega_8^4)$.
The remaining accumulators after $(x^2 + \omega_8^4)$ on this level are $\{(x^2 + \omega_8^0), (x^2 + \omega_8^6), (x^2 + \omega_8^2)\}$.
Recursing on the next level, its accumulators are $\langle(x^4 + \omega_8^4),(x^4 + \omega_8^0)\rangle$.
Finally, the root will be $(x^8-\omega_8^0) = (x^8 - 1) = \prod_{i=0}^{7}(x-\omega_8^i)$.
% bitreverse(000)=000=0
% bitreverse(001)=100=4
% bitreverse(010)=010=2
% bitreverse(011)=110=6
% bitreverse(100)=001=1
% bitreverse(101)=101=5
% bitreverse(110)=011=3
% bitreverse(111)=111=7

% bitreverse(00)=00
% bitreverse(01)=10
% bitreverse(10)=01
% bitreverse(11)=11
%
% (x-w_4^0)(x-w_4^2) = (x^2 - x w_4^0 -x w_4^2 - w_4^2) = (x^2 - x (w_4^0 + w_4^2) - w_4^2).
% recall that \omega_n^{k + n/2} = -\omega_n^k
% since n = 4, we have w_4^{0+2} = -w_4^0
% so we have (x^2 - w_4^2), which equals (x^2 + w_4^0) = (x^2 + 1)
%
% (x-w_4)(x-w_4^3) = (x^2 - x (w_4 + w_4^3) - w_4^4) = (x^2 - x (w_4 + w_4^3) - 1)
% recall that \omega_n^{k + n/2} = -\omega_n^k
% since n = 4, we have w_4^{1 + 2} = - w_4, i.e., w_4^3 = -w_4
% so we have (x^2 - 1)

\subsubsection{Do AMTs need extra public parameters?}
\label{s:amt:public-parameters}
%, which would impose unwanted overhead in the trusted setup phase (see \cref{s:trusted-setup})?
Recall that in KZG, given $(t-1)$-SDH public parameters, one can commit to any degree-bound $t$ polynomials and compute \textit{any number} of KZG evaluation proofs.
In contrast, computing an AMT at $n > t-1$ points seems to require committing to degree $n > t-1$ accumulator polynomials (e.g., to the root accumulator $(x^n-1)$).
Yet this is not possible given only $(t-1)$-SDH parameters, as ensured by the $(t-1)$-polyDH assumption (see \cref{s:assumptions}).
Fortunately, when computing an AMT, divisions by accumulators of degree $> t-1$ always give quotient zero (see \cref{s:amt:proof-time-and-sizes}).
This means that, when pairing such quotients with their accumulators during proof verification, the result will always be $1_{\Group_T}$ (see \cref{eq:amt-verify}).
In other words, such pairings need never be computed and so their corresponding accumulators (of degree $>t-1$) need never be committed to.
Furthermore, quotients are not problematic since they always have degree $< \deg{\phi} = t-1$ (or are equal to zero).

Second, AMT verifiers only need a logarithmic number of $g^{\tau^{2^k}}$ powers to recreate any accumulator commitment $g^{a_w(\tau)}$.
(This is a bit worse than KZG verifiers, who only need $g^\tau$.)
Specifically, given a subset $\{g^{\tau^{2^k}} \mathrel| 0 \le k\le \floor{\log(t-1)}\}$ of the $(t-1)$-SDH parameters, the verifier can commit to any degree-bound $t$ accumulator of the $(x^{2^k}+c)$ form.
Thus, we impose no additional overhead in the trusted setup. 
In contrast, if we evaluated $\phi$ at $\{1,2,\dots, n\}$, verifiers would need all $(t-1)$-SDH public parameters to reconstruct the accumulators.
