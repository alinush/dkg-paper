\subsection{AMT Prover Time and Proof Sizes}
\label{s:amt:proof-time-and-sizes}
We will restrict ourselves to our $n = 2^m$ and $\deg{\phi}=t-1 < n$ setting.
We first show that computing our optimized, roots-of-unity-based AMT takes $O(n\log{t})$ time (see \cref{s:amt:roots-of-unity}).
The key observation is that, when computing the AMT, divisions at higher levels (i.e., closer to the root) in the tree are \textit{trivial} and need not be performed.
Specifically, at sufficiently high levels, the degree of the divisors (i.e., accumulators) are larger than the degrees of the dividends (i.e., remainders), and always give quotients equal to zero.
Since zero quotients can be easily recreated by verifiers, their commitments need not be included in the proof.
We expand on this next.

Let us number levels differently, from $\log{n}$ (the root) to 0 (the leaves), so that level $i$ has $n/2^i$ nodes, each with an accumulator of degree $2^i$.
Now, let $k$ be the smallest value such that $2^k \le \deg{\phi} < 2^{k+1}$.
In other words, $k$ is the level at which accumulator degrees are $\le \deg{\phi}$ and thus divisions are non-trivial.
Put differently, each node on level $k$ will be the root node of an (authenticated) multipoint evaluation (sub)tree.
We argue that the time to compute any one such subtree is $O(2^k \log{2^k})$ and, since there are $n/2^k$ such subtrees, the final AMT takes $O(n\log{2^k}) = O(n\log{t})$ time since $2^k\le t-1=\deg{\phi}$.
We prove this inductively next.

At the root node of a level $k$ subtree, the dividend $d_k = \phi$ has $\deg{d_k} < 2^{k+1}$ (by definition of $k$ above).
The accumulator $a_k$ has $\deg{a_k} = 2^k$.
Thus, the quotient $q_k = d_k/a_k$ will have $\deg{q_k} = \deg{d_k}-\deg{a_k}< 2^{k+1} - 2^k = 2^k$ and the remainder $r_k = d_k \bmod a_k$ will have $\deg{r_k} < \deg{a_k} = 2^k$.
The division at this level will only take $O(\deg{d_k})=O(2^{k+1})$ time, thanks to the $(x^{2^k} + c)$ form of $a_k$.
Committing to the quotient will take $O(2^{k})$ time.
To summarize, at level $k$ we are doing $O(2^{k+1})$ work and $\deg{d_k} < 2^{k+1}, \deg{a_k} = 2^k, \deg{q_k} < 2^k, \deg{r_k} < 2^k$.
Next, we argue that the amount of work \textit{per node} on level $k-1$ is half the work per node at level $k$.
This is because (1) the dividend $d_{k-1}$ is set to the remainder $r_k$ from the parent, so $\deg{d_{k-1}} < 2^k$, (2) $\deg{a_{k-1}}=2^{k-1}$, (3) $\deg{q_{k-1}} = \deg{d_{k-1}}-\deg{a_{k-1}}<2^k-2^{k-1}=2^{k-1}$ and (4) $\deg{r_{k-1}} < \deg{a_{k-1}}=2^{k-1}$.
Thus, at level $k-1$, the division takes $O(2^k)$ time and committing to the quotient takes $O(2^{k-1})$ time. 
As a result, the time to compute the subtree can be expressed as $T(2^{k+1}) = 2T(2^{k+1} / 2) + O(2^{k+1}) = O(2^{k}\log{2^{k}})$.

% - Q: What is proof computation time then? A: $O(n\log{t})$
%    - Let $k$ be the greatest value such that $2^k \le t-1 < 2^{k-1}$.
%    - $k$ is the level where we actually start dividing the root polynomial $p$ of degree $t-1$ by an accumulator of degree $2^k$
%    - At level $k$ (where $k=0$ are the leaves), we have $n/2^k$ nodes.
%    - At each node on level $k$, we do an $O(t)$ division (dividend has deg $t-1$ and divisor has deg less than that)
%    - And, in that node's subtree, we do $O(t\log{t})$ work overall
%    - So, in the whole tree we do $O(n/2^k t\log{t})$ work
%    - So close to $O(n\log{t})$ time

Finally, an AMT proof is $O(\log{t})$-sized.
Recall that quotients in the AMT are non-zero only at levels $k$ and below, where $2^k \le t-1 < 2^{k+1}$.
Thus, an AMT proof will only have non-zero quotients at levels $k, k-1, k-2, \dots, 1, 0$.
Since $k = \floor{\log_2(t-1)}$ the exact proof size is $\floor{\log_2(t-1)}+1$ group elements.

% - Q: Are our proofs $O(\log{t})$ or $O(\log{n})$?
% - A: $\mathsf{floor}(\log_2{(t-1)}) + 1$
%    + We only start dividing by accumulators of deg $< t$
%    - Each time we divide, the next level has an accumulator of deg $t/2$ and a remainder of deg $< t-1$
%    - $\log_2{(t-1)}$ makes sense because the accumulators from the leaves to the roots have degree $1, 2, 4, \dots, n$, at some point reaching a maximum degree $2^k \le t-1 \Leftrightarrow k \le \log_2{(t-1)}$
%       + i.e., $k = \mathsf{floor}(\log_2(t-1))$
%    - The exact proof size should be $k+1$, I think. For example, consider $t-1 = 2$ (or even 3). 
%        - Then, the largest accumulator we'll "successfully" / "usefully" divide by will be deg $2^k = 2^1$. (Larger accumulators will give quotient 0 and remainder equal to the dividend.) 
%        - Dividing by the acc of deg $2^1$ will give a remainder of deg $\le$ 1 and a non-zero quotient.
%        - Another division will be needed at the level below, by an accumulator of deg 1.
%        - Thus, the proof consists of these two quotients.
%        - So its size is $k+1$.
%    - So in the code, the proof size should be $1 + \mathsf{floor}(\log_2(t-1))$
