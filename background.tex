\section{Preliminaries}
\label{s:prelim}
In this section we introduce some notation, our cryptographic assumptions and the communication and adversarial model for the distributed protocols in this paper.
Then, we give background on TSS, polynomial commitments, VSS, DKG and polynomial multipoint evaluations.

\subsubsection{Notation}
\label{s:prelim:notation}
Let $\Fp$ denote the finite field ``in the exponent'' associated with a group $\Group$ of prime order $p$ with generator $g$.
We use multiplicative notation for all algebraic groups in this paper.
Let $1_{\Group}$ denote the identity element of a group $\Group$.
Let $s \in_R S$ denote sampling an element $s$ uniformly at random from some set $S$.
Let $\log{x}$ be shorthand for $\log_2{x}$.
Let $[i,j] = \{i,i+1,\dots,j-1,j\}$ and $[n] = [1,n]$ and.
Let $\deg{\phi}$ denote the degree of a polynomial $\phi$.
We say a polynomial $\phi$ has \textit{degree-bound} $m$ if $\deg{\phi} < m$.
Let $\PP_q(g;\tau) = \langle g^\tau, g^{\tau^2}, \dots, g^{\tau^q}\rangle$ denote public parameters used in the $q$-SBDH assumption (see \cref{s:assumptions}).

\subsubsection{Cryptographic assumptions}
\label{s:prelim:cryptographic-assumptions}
Our work relies on the use of \emph{pairings} or \emph{bilinear maps}, first introduced by Menezes et al. \cite{mov-attack}.
Recall that a bilinear map $e(\cdot,\cdot) : \Group \times \Group \rightarrow \Group_T$ has useful algebraic properties: $e(g^a, g^b) = e(g^a, g)^b = e(g, g^b)^a = e(g, g)^{ab}$.
To simplify exposition, throughout the paper we assume symmetric (Type~I) pairings, but our results can be re-stated in the setting of the more efficient asymmetric (Type II and III) pairings in a straightforward manner.
Our schemes from \cref{s:scalable-vss,s:scalable-dkg} rely on the $q$-SBDH~\cite{qsbdh} and $q$-polyDH~\cite{polycommit} assumptions, both defined in \cref{s:assumptions}.

\subsubsection{Communication and adversarial model}
DKG and VSS protocols assume a \textit{broadcast channel} for all actors to reliably communicate with each other~\cite{CGMA85,Pedersen1991AThreshold}.
(In practice, this can be implemented using BFT protocols~\cite{SJSW19}.)
In addition, some protocols need \textit{private and authenticated channels} between actors~\cite{Feldman1987Practical,Pedersen1991AThreshold,polycommit,dkg,Kate2010Distributed}.
We focus on \textit{synchronous} VSS and DKG protocols, where parties communicate in \textit{rounds}.
Within a round, each party performs some computation, (possibly) sends private messages to other players and broadcasts a message to everybody.
By the end of the round, each party receives all messages sent in that round by other players (whether privately or via broadcast).
We assume computationally-bounded adversaries \Adv that control up to $t-1$ players.
%In other words, we only focus on \textit{computationally-secure} VSS and DKG protocols, since \textit{unconditional security} is inherently costlier and more difficult to scale.
We restrict ourselves to \textit{static} \Adv's who fix the set of $<t$ corrupted players before the protocol starts.
% Note: FGG+06 is one VSS paper that talks about rushing adversaries.
We assume \Adv can be \textit{rushing} and can wait to hear all messages from all honest players in a round before privately sending or broadcasting his own message within that same round.
% Note: Why you do NOT need an honest majority to disqualify dishonest players in a DKG protocol:
%  - if t-1 players say player i is bad, then they could all be malicious, so you can't disqualify i
%  - it t   players say player i is bad, then one of the players must have been honest, so I we can safely disqualify player i.
The protocols in this paper are \textit{robust}: there are always $t$ honest players who can reconstruct the secret.
In the synchronous setting, robustness holds for all $t - 1 < n/2$~\cite{dkg}.
%To see this, note the minimum number of honest players is $h = n - (t-1)$, since an adversary can compromise up to $t-1$ players.
%Since $t$ honest are needed to reconstruct, this is the same thing as saying $h \ge t\Leftrightarrow h > t-1 \Leftrightarrow n-(t-1) > t-1 \Leftrightarrow n > 2t-2 \Leftrightarrow n/2 > t-1$.

\subsubsection{FFT and Lagrange interpolation}
\label{s:prelim:fft}
% moderncomputeralgebra-ch8 talks about FFT in finite field (roots of unity to coefficients)
% moderncomputeralgebra-ch9 talks about fast division using Newton iteration
% moderncomputeralgebra-ch10 talks about evaluation and interpolation using FFT at arbitrary points: x_i, f(x_i)
% moderncomputeralgebra-ch11 talks about fast Euclidean Algorithm (Theorem 11.10)
We use the \textit{Fast Fourier Transform (FFT)} to multiply and divide polynomials in $\Fp[X]$ of degree-bound $N = 2^k$ in $\Theta(N\log{N})$ time~\cite{CLRS09,moderncomputeralgebra-ch9}.
For this, we need a primitive $N$th \textit{root of unity} in $\Fp$, which we denote by $\omega_N$~\cite{CLRS09}.
% Recall that an FFT of size $N$ on a polynomial $\phi=(c_0, c_1, c_2, \dots, c_n)$ of degree $n<N$ computes $(\phi(\omega^i))_{i\in[0,n-1]}$ in $\Theta(n\log{n})$ field operations.
%
% Aho, Ullman "The Design and Analysis of Computer Algorithms" book says in Section 8.3 "Polynomial multiplication and divison", Theorem 8.7 that multiplication and division of any polynomial of degree n take the same time.
%
% Preparata paper: Remark at the end states that convolutions of any size can be computed. In other words, polynomials of any degree can be multiplied.
%
% Note: In our implementation, we never need to use FFT-based division: 
% (1) Our AMT-based protocols only divide by (x^{2^i} + \omega_n^k) polynomials, which can be done in O(2^i) time
% (2) Our BLS TSS uses an FFT rather than a multipoint eval to evaluate the N'(x) derivative at all roots-of-unity, discarding unnecessary values
% However, in the paper, many algorithms require fast division (e.g., multipoint eval, fast Lagrange, KZG batch proofs), so we must mention it is doable in n\log{n} time.
Finally, given $(x_i, y_i = \phi(x_i))_{i\in[n]}$, interpolating $\phi$ takes $\Theta(n\log^2{n})$ time using \textit{fast Lagrange interpolation}~\cite{moderncomputeralgebra-ch10}.
Specifically, recall that $\phi(x)=\sum_{i\in[n]} \Ell_i^{[n]}(x) y_i$ where $\Ell_i^{[n]}(x)=\prod_{{j\in[n]\\j\ne i}} \frac{x-x_j}{x_i-x_j}$ is called a \textit{Lagrange polynomial}~\cite{BT04}.
Note that $\Ell_i^{[n]}$ is defined with respect to the set of points $\{x_i\}_{i\in[n]}$.
Throughout this paper, this set will typically be $\{x_i\}_{i\in T}$ where $T\subset [n]$, with either $x_i = i$ or $x_i = \omega_N^{i-1}$ and the Lagrange polynomial will be denoted $\Ell_i^T(x)$.


%\newcommand{\showdiv}[2]{\makecell*[{{p{5cm}}}]{#1\\#2}}
%\newcommand{\showdiv}[2]{\makecell{#1\\#2}}
%\newcommand{\showdiv}[2]{\makecell{#1\\*#2}}
%\newcommand{\showdiv}[2]{\makecell{#1\linebreak#2}}
%\newcommand{\showdiv}[2]{\makecell{#1\newline#2}}
\newcommand{\showdiv}[2]{#2}
%\newcommand{\showdiv}[2]{#1}
\newcommand{\quoNode}[1]{\myred{#1}}
\newcommand{\leafRem}[1]{\myblue{\mathbf{#1}}}

\newcommand{\multipointEvalFigure}{
\begin{figure*}[t]
{
\scriptsize
%\small
\begin{forest}
for tree={
%    fit=band,% spaces the tree out a little to avoid collisions
%    fit=tight,% spaces the tree out less
%    fit=rectangle,
    inner sep=4,
}
[{\showdiv{$r_{1,8} = \phi \bmod (x-1)(x-2)\dots(x-8)$}{$\phi = \quoNode{q_{1,8}} (x-1)(x-2)\dots(x-8) + r_{1,8}$}}
    [{\showdiv{$r_{1,4} = r_{1,8} \bmod (x-1)(x-2)\dots(x-4)$}{$r_{1,8} = \quoNode{q_{1,4}} (x-1)(x-2)\dots(x-4) + r_{1,4}$}}
        [{\showdiv{$r_{1,2} = r_{1,4} \bmod (x-1)(x-2)$}{$r_{1,4} = \quoNode{q_{1,2}} (x-1)(x-2) + r_{1,2}$}}
            [{\showdiv{$r_{1,1} = r_{1,2} \bmod (x-1)$}{$r_{1,2} = \quoNode{q_{1,1}} (x-1) + \leafRem{r_{1,1}}$}}
                [, no edge, tier=odd ]
            ]
            [{\showdiv{$r_{2,2} = r_{1,2} \bmod (x-2)$}{$r_{1,2} = \quoNode{q_{2,2}} (x-2) + \leafRem{r_{2,2}}$}}, tier=odd
            ]
        ]
        [{\showdiv{$r_{3,4} = r_{1,4} \bmod (x-3)(x-4)$}{$r_{1,4} = \quoNode{q_{3,4}}(x-3)(x-4) + r_{3,4}$}}
            [{\showdiv{$r_{3,3} = r_{3,4} \bmod (x-3)$}{$r_{3,4} = \quoNode{q_{3,3}}(x-3) + \leafRem{r_{3,3}}$}}
                [, no edge, tier=odd ]
            ]
            [{\showdiv{$r_{4,4} = r_{3,4} \bmod (x-4)$}{$r_{3,4} = \quoNode{q_{4,4}}(x-4) + \leafRem{r_{4,4}}$}}, tier=odd
            ]
        ]
    ]
    [{\showdiv{$r_{5,8} = r_{1,8} \bmod (x-5)(x-6)\dots(x-8)$}{$r_{1,8} = \quoNode{q_{5,8}} (x-5)(x-6)\dots(x-8) + r_{5,8}$}} 
        [{\showdiv{$r_{5,6} = r_{5,8} \bmod (x-5)(x-6)$}{$r_{5,8} = \quoNode{q_{5,6}}(x-5)(x-6) + r_{5,6}$}} 
            [{\showdiv{$r_{5,5} = r_{5,6} \bmod (x-5)$}{$r_{5,6} = \quoNode{q_{5,5}}(x-5) + \leafRem{r_{5,5}}$}}
                [, no edge, tier=odd ]
            ]
            [{\showdiv{$r_{6,6} = r_{5,6} \bmod (x-6)$}{$r_{5,6} = \quoNode{q_{6,6}}(x-6) + \leafRem{r_{6,6}}$}}, tier=odd
            ]
        ]
        [{\showdiv{$r_{7,8} = r_{5,8} \bmod (x-7)(x-8)$}{$r_{5,8} = \quoNode{q_{7,8}}(x-7)(x-8) + r_{7,8}$}} 
            [{\showdiv{$r_{7,7} = r_{7,8} \bmod (x-7)$}{$r_{7,8} = \quoNode{q_{7,7}}(x-7)+\leafRem{r_{7,7}}$}}
                [, no edge, tier=odd ]
            ]
            [{\showdiv{$r_{8,8} = r_{7,8} \bmod (x-8)$}{$r_{7,8} = \quoNode{q_{8,8}}(x-8)+\leafRem{r_{8,8}}$}}, tier=odd
            ]
        ]
    ]
]
\end{forest}
}

\caption{A multipoint evaluation of polynomial $\phi$ at points $[8]=\{1,2,\dots, 8\}$.
    Each node is expressed as $a = q \cdot b + r$: i.e., a polynomial $a$ is being divided by $b$, resulting in a \textit{quotient} $q$ and a \textit{remainder} $r$.
    In the root node, $\phi$ is divided by the root \textit{accumulator} $\prod_{i\in[8]}(x-i)$, obtaining a quotient $q_{1,8}$ and a remainder $r_{1,8}$.
    Then, the root's left child divides $r_{1,8}$ by $(x-1)\cdots(x-4)$ while the right child divides it by $(x-5)\cdots(x-8)$.
    The process is repeated recursively on the resulting $r_{1,4}$ and $r_{5,8}$ remainders.
    The remainders $r_{i,i}$ in the leaves are the evaluations $\phi(i)$.}
\label{f:multipoint-eval}
\end{figure*}
}

\subsection{Threshold Signature Schemes (TSS)}
\label{s:prelim:threshsig}

A $(t,n)$-threshold signature scheme (TSS) is a protocol amongst $n$ \textit{signers} where \textit{only} subsets of size $\ge t$ can produce a \textit{digital signature}~\cite{rsa} on a message $m$.
Many signature schemes can be turned into a TSS, such as RSA~\cite{rsa,Shoup2000Practical}, Schnorr~\cite{Schnorr89,threshold-schnorr,Gennaro2003SecureApplications}, ElGamal~\cite{ElGamal1985APublicKey,Harn1994GroupOriented,Park96NewElGamal,Gennaro1996Robust}, ECDSA~\cite{GennaroGoldfederNarayanan2016ThresholdOptimal} and BLS~\cite{bls,Boldyreva2003Threshold}.
In this paper, we focus on the BLS TSS because of its simplicity and efficiency.

\subsubsection{(Threshold) BLS signatures}
A normal BLS signature on a message $m\in \{0,1\}^*$ is $\sigma = H(m)^s$ where $s\in_R \Fp$ is the \textit{secret key} and $H : \{0,1\}^* \rightarrow \Group$ is a hash function modeled as a random oracle.
To verify the signature against the \textit{public key} $g^s$, a bilinear map $e$ is used to ensure that $e(H(m), g^s) \stackrel{?}{=} e(\sigma, g)\Leftrightarrow e(H(m),g)^s \stackrel{?}{=} e(H(m)^s, g)$.

To obtain a $(t,n)$ BLS TSS~\cite{Boldyreva2003Threshold}, the secret key $s$ is split amongst the $n$ signers using $(t,n)$ Shamir secret sharing (see \cref{s:prelim:shamir-secret-sharing}).
Specifically, each signer $i$ has a \textit{secret key share} $s_i$ of $s$ along with a \textit{verification key} $g^{s_i}$.
%Reconstructing $s$ directly and then using it to sign $m$ would only result in a one-time threshold signature scheme, since once $s$ is revealed to a signer, the scheme is no longer a threshold scheme.
To produce a signature on $m$, each $i$ computes a \textit{signature share} $\sigma_i = H(m)^{s_i}$.
Then, all $\sigma_i$'s are sent to an \textit{aggregator} (e.g., one of the signers).
Since some signers are malicious, their $\sigma_i$ might not be valid.
Thus, the aggregator verifies each $\sigma_i$ by checking if $e(g^{s_i}, H(m)) \stackrel{?}{=} e(\sigma_i, g)$.
(This works because $\sigma_i$ is a normal BLS signature that should verify under $g^{s_i}$.)
This way, the aggregator finds a subset $T$ of $t$ signers who produced a valid signature share $\sigma_i$.
Now, the aggregator can compute the final signature as $\sigma = \prod_{i\in T} {\sigma_i^{\Ell_i^T(0)}} = H(m)^{\sum_{i\in T} {s_i \Ell_i^T(0)}} = H(m)^s$ via Lagrange interpolation (see \cref{s:prelim:fft}).
Importantly, aggregation never exposes the secret key $s$, which is interpolated ``in the exponent.''
The time to \textit{aggregate} the signature is $\Theta(t^2)$, dominated by the time to (naively) compute the $\Ell_i^T(0)$'s.

\subsection{Constant-sized Polynomial Commitments}
\label{s:prelim:polycommit}
Kate, Zaverucha and Goldberg introduced constant-sized polynomial commitments, often called \textit{KZG commitments}~\cite{polycommit}.
Their scheme requires $\ell$-SDH~\cite{BonehBoyen2008} public parameters $\PP_\ell(g;\tau)=(g^{\tau^i})_{i\in[0,\ell]}$ where $\tau$ denotes a trapdoor.
(These parameters are computed via a \textit{trusted setup}; see \cref{s:trusted-setup}.)
Their scheme is \textit{computationally-hiding} (see \cref{d:polycommit:comp-hiding}) under the discrete log assumption and \textit{computationally-binding}~\cite{polycommit} under $\ell$-SDH.
Unlike Pedersen commitments~\cite{Pedersen1991Noninteractive}, KZG can only commit to polynomials of maximum degree $\ell$.

Let $\phi$ denote a polynomial of degree $d\le \ell$ with coefficients $c_0, c_1, \dots, c_d$ in $\Fp$.
A KZG commitment to $\phi$ is a single group element $C = \prod_{i=0}^d {\left(g^{\tau^i}\right)^{c_i}} = g^{\sum_{i=0}^d c_i \tau^i} = g^{\phi(\tau)}$.
Note that committing to $\phi$ takes $\Theta(d)$ time.
To compute an \textit{evaluation proof} that $\phi(a) = y$, KZG leverages the polynomial remainder theorem, which says:
\begin{align}
    \phi(a) = y \Leftrightarrow \exists q, \phi(x) - y = q(x)(x-a)
    \label{eq:poly-rem-thm}
\end{align}
The proof is just a KZG commitment to $q$: a single group element $\pi=g^{q(\tau)}$.
Computing the proof takes $\Theta(d)$ time.
To verify $\pi$, one checks (in constant time) if 
$e(C / g^y, g) = e(\pi, g^{\tau}/g^a)\Leftrightarrow e(g,g)^{\phi(\tau)-y} = e(g,g)^{q(\tau)(\tau-a)}$.
%\begin{align*}
%e(C / g^y, g)            &\stackrel{?}{=} e(g^{\pi}, g^{\tau}/g^a) \Leftrightarrow\\
%e(g^{\phi(\tau) - y}, g) &\stackrel{?}{=} e(g^{q(\tau)},g^{\tau-a})\Leftrightarrow\\
%e(g,g)^{\phi(\tau)-y}    &\stackrel{?}{=} e(g,g)^{q(\tau)(\tau-a)} \Leftrightarrow\\
%{\phi(\tau)-y}           &\stackrel{?}{=} q(\tau)(\tau-a)
%\end{align*}

\subsubsection{Batch proofs and homomorphism}
\label{s:prelim:polycommit:kzg:batch}
\label{s:prelim:polycommit:kzg:homomorphism}
Given a set of points $S$ and their evaluations $\{\phi(i)\}_{i\in S}$, KZG can prove all evaluations with \textit{one} constant-sized \textit{batch proof} rather than $|S|$ individual proofs~\cite{polycommit}.
The prover computes an \textit{accumulator polynomial} $a(x)=\prod_{i\in S} (x-i)$ in $\Theta(|S|\log^2{|S|})$ time and computes $\phi/a$ in $\Theta(d\log{d})$ time, obtaining a quotient $q$ and remainder $r$.
The batch proof is $\pi=g^{q(\tau)}$.
To verify $\pi$ against $\{\phi(i)\}_{i\in S}$ and $C$, the verifier first computes $a$ from $S$ and interpolates $r$ such that $r(i)=\phi(i), \forall i \in S$ in $\Theta(|S|\log^2{|S|})$ time.
Next, he computes $g^{a(\tau)}$ and $g^{r(\tau)}$ commitments.
Finally, he checks if $e(C / g^{r(\tau)}, g) = e(g^{q(\tau)}, g^{a(\tau)})$.
We stress that batch proofs are only useful when $|S| \le d$.
Otherwise, if $|S| > d$, we can interpolate $\phi$ directly from the evaluations, which makes verifying any evaluation trivial.

Finally, KZG proofs have a \textit{homomorphic} property.
Suppose we have two polynomials $\phi_1, \phi_2$ with commitments $C_1,C_2$ and two proofs $\pi_1,\pi_2$ for $\phi_1(a)$ and $\phi_2(a)$, respectively.
Then, a commitment $C$ to the sum polynomial $\phi=\phi_1 + \phi_2$ can be computed as $C=C_1 C_2=g^{\phi_1(\tau)} g^{\phi_2(\tau)} = g^{\phi_1(\tau) + {\phi_2(\tau)}} = g^{(\phi_1 +\phi_2)(\tau)}$.
Even better, a proof $\pi$ for $\phi(a)$ w.r.t. $C$ can be aggregated as $\pi = \pi_1 \pi_2$.
This homomorphism is necessary in KZG-based protocols such as \ejfdkg (see \cref{s:prelim:dkg}).

% Note: "hiding" for degree t polynomial \phi() is defined as:
% - Adversary gets t evaluations \phi(x_i) \textbf{and} proofs \pi_i, i\in[t] and has to guess \phi(x*) for any x* != x_i.
% Unconditional hiding means no information about \phi(x*) can be determined by adversary.
% Computational hiding means no PPT adversary can output \phi(x*).
%
% PolyCommit_DL is NOT information-theoretic hiding because g^{\phi(\tau)} is exposed, so information about $\phi(x*)$ at $x* = \tau$ is revealed.
%
%
% Q: An interesting question is if, given t \phi(x_i)'s, can we interpolate g^{\phi(x*)} in the exponent?
% A: I would need to compute Lagrange coefficients \ell_i = \Ell_i^T(x*) = \prod_{j\in [0,t],  j\ne i} (x* - x_j)/(x_i - x_j) where {x_k}_{k \in [t]} are the extra t points and x_0 = \tau
% But I cannot do this when I don't know \tau, because the denominator needs to have \prod_{j\in[t]}(\tau - x_j).
% Really, I need to just compute $\ell_0$ (to exponentiate g^{\phi(\tau)} with) and then I only need {g^{\ell_i}}_{i\in[t]} (to exponentiate them with \phi(x_i)).
%   - but I can't compute \ell_0 because its denominator is \prod_{j\in[1,t]} (\tau - x_j)
%   - and I can't compute \ell_i with i \ne 0 because it has an (x* - \tau) in the numerator
%
% What if I interpolate at a particular x* rather than an arbitrary one?
% I need an x* that makes computing the \ell_i's easier?
% I have the quotient commitments, so maybe that helps?

\newcommand{\evssAlgorithm}{
\setlist[enumerate]{leftmargin=15pt, itemsep=.3pt}
\setlist[itemize]{leftmargin=*, itemsep=.3pt}
\begin{algorithm}[t] % single column algorithm
    \caption{\small \evss: A synchronous $(t,n)$ VSS}
    \label{alg:vss}
    \footnotesize
    
    \vspace{.2em}
    \begin{center}
        \textbf{Sharing Phase}
    \end{center}
    \vspace{-.5em}
    \underline{Dealing round:}
    \begin{enumerate}
        \item The dealer picks $\phi\in_R \Fp[X]$ of degree $t-1$ with $s = \phi(0)$, computes all shares $s_i = \phi(i)$, and commits to $\phi$ as $c = g^{\phi(\tau)}$. 
        \item Computes KZG proofs $\pi_i = g^{q_i(\tau)}$, $q_i(x) = \frac{\phi(x)-\phi(i)}{x-i}$, $\forall i\in[n]$.
        \item \textit{Broadcasts} $c$ to all players. Then, sends $(s_i, \pi_i)$ to each player $i\in[n]$ over an \textit{authenticated, private} channel.
    \end{enumerate}
    \underline{Verification round:}
    \begin{enumerate}
        \item Each player $i\in [n]$ verifies $\pi_i$ against $c$ by checking if $e(c / g^{s_i}, g) = e(\pi_i, g^{\tau-i})$.
              If this check fails (or $i$ received nothing from dealer), then $i$ broadcasts a \textit{complaint} against the dealer.
    \end{enumerate}
%    \underline{Complaint round (w/ KZG batch proofs):}
%    \begin{enumerate}
%        \item If the size of the set $S$ of complaining players is $\ge t$, then the dealer is \textit{disqualified}.
%              Otherwise, the dealer computes a \textit{KZG batch proof} $\pi$ for all $\{s_i\}_{i\in S}$ and broadcasts $(S, \{s_i\}_{i\in S}, \pi)$.
%              % In some sense, the round has finished after the KZG batch proof was broadcast, and the verification work done below
%              % should be considered as part of a new round. But since this verification is not followed by another broadcast, it is not considered as a new round.
%        \item If the batch proof does not verify, then the dealer is disqualified. Otherwise, each $i\in S$ now has his correct share $s_i$.
%    \end{enumerate}
    \underline{Complaint round:}
    \begin{enumerate}
        \item If the size of the set $S$ of complaining players is $\ge t$, the dealer is \textit{disqualified}.
              Otherwise, the dealer reveals the correct shares with proofs by broadcasting $\{s_i, \pi_i\}_{i\in S}$.
              % In some sense, the round has finished after the KZG batch proof was broadcast, and the verification work done below
              % should be considered as part of a new round. But since this verification is not followed by another broadcast, it is not considered as a new round.
        \item If any one proof does not verify (or dealer did not broadcast), the dealer is disqualified. Otherwise, each $i\in [n]$ now has his correct share $s_i$.
    \end{enumerate}
    
    \begin{center}
        \textbf{Reconstruction Phase}
    \end{center}
    \vspace{-.5em}
    Given commitment $c$ and shares $(i, s_i,\pi_i)_{i\in T}$, $|T|\ge t$, the \textit{reconstructor}:
    \begin{enumerate}
        \item Verifies each $s_i$, identifying a subset $V$ of $t$ players with valid shares.
        \item Interpolates $s=\sum_{i\in V} \Ell_i^V(0) s_i=\phi(0)$. 
    \end{enumerate}
\end{algorithm}
}

\evssAlgorithm

\subsection{(Verifiable) Secret Sharing}
\label{s:prelim:vss}
\label{s:prelim:shamir-secret-sharing}
A $(t,n)$ \textit{secret sharing} scheme allows a \textit{dealer} to split up a secret $s$ amongst $n$ \textit{players} such that \textit{only} subsets of size $\ge t$ players can reconstruct $s$.
Secret sharing schemes were introduced independently by Shamir~\cite{how-to-share-a-secret} and Blakley~\cite{Blakley1979}.
%In this paper, we focus on Shamir's secret sharing (SSS) protocol and its extensions.
\textit{Shamir's secret sharing (SSS)} is split into two phases.
In the \textit{sharing phase}, the dealer picks a degree $t-1$, random, univariate polynomial $\phi$, lets $s=\phi(0)$ and distributes a \textit{share} $s_i = \phi(i)$ to each player $i\in [n]$.
In the \textit{reconstruction phase}, any subset $T\subset [n]$ of $t$ honest players can reconstruct $s$ by sending their shares to a \textit{reconstructor}.
For each $i\in T$, the reconstructor computes a \textit{Lagrange coefficient} $\Ell_i^T(0) = \prod_{j\in T, j\ne i} {\frac{0-j}{i - j}}$.
Then, he computes the secret as $s = \phi(0) = \sum_{i\in T} \Ell_i^T(0) s_i$ (see \cref{s:prelim:fft}).


% Q: What if dealer picks polynomial of degree less than $t-1$ and commits to it? The Kate et al. commitment doesn't reveal the degree.
% A: The answer is that secrecy is only guaranteed for *honest* dealers, since a dishonest dealer can always give some of the players the secret.
% 
% Q: What if dealer picks polynomial of degree t or higher? Then different subsets of t players will reconstruct different secrets.
% A: Dealer can't do this because of the $(t-1)$-polyDH assumption.
Unfortunately, SSS does not tolerate malicious dealers who distribute invalid shares, nor malicious players who might send invalid shares during reconstruction.
To deal with this, \textit{Verifiable Secret Sharing (VSS)} protocols enable players to verify shares from a potentially-malicious dealer~\cite{CGMA85,Feldman1987Practical,Pedersen1991Noninteractive,polycommit}.
Furthermore, VSS also enables the reconstructor to verify the shares before interpolating the (wrong) secret.
Loosely speaking, VSS protocols must offer two properties against any adversary who compromises the dealer and $<t$ players: \textit{secrecy} and \textit{correctness}.
Secrecy guarantees that no adversary learns the secret $s$ when the dealer is honest, since a malicious one can simply reveal $s$.
Correctness guarantees that, after the sharing phase, either any set of $\ge t$ honest players can always reconstruct $s$ or the dealer is \textit{disqualified}.
We refer the reader to~\cite{polycommit} for more formal VSS definitions. %, since our paper extends Kate et al.'s \evss.

% Typically the round complexity of VSS means the round complexity of the sharing phase:
% For example, \evss has 1 round for the dealer to broadcast the commitment and send shares.
% Then, another round for the players to send their complaints.
% Then, another round for the dealer to send back a batch proof for all complaining players.
% In ~\cite{BKP11}, a 2-round VSS is presented, surprisingly.

\subsubsection{Kate et al.'s \evss}
\label{s:prelim:evss}
At a high-level, \evss follows the style of previous VSS protocols~\cite{Feldman1987Practical,Pedersen1991Noninteractive}.
In the \textit{dealing round}, the dealer commits to $\phi$ and sends each player their share and \textit{proof} that their share is correct.
In the \textit{verification round}, each player verifies the proof for his share and, if incorrect, broadcasts a \textit{complaint}.
Finally, in the \textit{complaint round}, the dealer resolves complaints (if any) by broadcasting the correct share of each complaining player.
We give a detailed description of \evss in \cref{alg:vss} and its asymptotic complexity in \cref{t:vss-comparison}.

From \cref{alg:vss}, \evss's \textit{overall communication} complexity is $\Theta(n)$ (since at most $2n+(t-1)$ shares and proofs are sent while dealing, complaining and reconstructing).
\evss's reconstruction phase is $O(t\log^2{t} + n)$ time, since at most $n$ shares have to be verified before the secret can be interpolated fast in $\Theta(t\log^2{t})$ time~\cite{moderncomputeralgebra-ch10}.
\evss's dealing round is $\Theta(nt)$ time, since $n$ KZG proofs must be computed.
% Note: We point out in the Evaluation section that the evaluations $\phi(i)$ are obtained ``for free'' during this proof computation.
% (It does not matter in terms of asymptotics, just concrete performance.)
The verification round is $\Theta(1)$ time (per player).
If $S$ is the set of complaining players, the complaint round takes $\Theta(|S|)$ time and communication for the dealer to send $|S|$ shares with proofs and $\Theta(|S|)$ time for each player to verify them.

% Note: In practice, \evss and Feldman's VSS~\cite{Feldman1987Practical} take the same amount of overall \textbf{asymptotic} time, because \evss just shifts the high verification cost of Feldman's VSS into the dealing phase.
% During reconstruction, for example, Feldman takes $\Theta(nt)$.
% But Feldman can be implemented on faster curves, so it is faster.

\newcommand{\ejfdkgAlgorithm}{
%\begin{multicols}{2}
%\begin{algorithm*} % two column algorithm
\setlist[enumerate]{leftmargin=*, itemsep=.3pt}
\setlist[itemize]{leftmargin=*, itemsep=.3pt}
\begin{algorithm}[t] % single column algorithm
    \caption{\small \ejfdkg's {Sharing Phase}}
    \label{alg:dkg}
    \footnotesize
        \underline{Dealing round:}
        Each player $i$:
        \begin{enumerate}
            \item Picks $f_i \in_R \Fp[X]$ of degree $t-1$, sets $z_i = f_i(0)$ and $c_i = g^{f_i(\tau)}$.
            % Note: A NIZKPoK of $z_i$ w.r.t. to $g^{z_i}$ is necessary for securely verifying the KZG proof against $g^{z_i}$.
            \item Computes $g^{z_i} = g^{f_i(0)}$, a KZG proof $\pi_{i,0}$ for $f_i(0)$ and a NIZKPoK $\nizkpok_i$ for $g^{f_i(0)}$
                  and \textit{broadcasts} $(c_i, g^{z_i}, \pi_{i,0}, \nizkpok_i)$.
            % NOTE: Sending $g^{z_i}$ could be deferred to later, and that would avoid the biasability of the final secret $s$ at the cost of an extra broadcast round. But PolyCommit_Ped must be used to hide g^{f_i(0)}.
            \item Computes shares $s_{i,j} = f_i(j)$ and KZG proofs $\pi_{i,j}$ and 
                  sends $(s_{i,j}, \pi_{i,j})$ to each $j\in[n]$ over an \textit{authenticated, private} channel.
        \end{enumerate}
        
        \underline{Verification round:} 
        For each $(c_i, g^{z_i}, \pi_{i,0}, \nizkpok_i, s_{i,j}, \pi_{i,j})$ from $i$, each $j$:
        \begin{enumerate}
            \item Verifies $\pi_{i,0}$ by checking $e(c_i/g^{z_i},g) = e(\pi_{i,0}, g^{\tau-0})$ and verifies the $\nizkpok_i$ NIZKPoK.
            % Note: The fact that g^{z_i} partially leaks the MSB of z_i is problematic here, because now the the adversary can disqualify a P_i that he controls if that favors a certain MSB in the final \sum_i g^{z_i}. He can disqualify P_i by having another controlled player P_j complain about P_i and then having P_i intentionally broadcast a bad share.
            \item Verifies its share $s_{i,j}$ using $e(c_i/g^{s_{i,j}},g) = e(\pi_{i,j}, g^{\tau-j})$.
            % Note: If the f_i(0) check fails for player j, it fails for all players => all honest players complain => player i is automatically disqualified
            \item If any of these checks fail (or nothing was received from $i$), then $j$ broadcasts a complaint against $i$.
        \end{enumerate}
        
%        \underline{Complaint round (w/ KZG batch proofs):}
%        \begin{enumerate}
%            \item Let $S_i$ be the set of players complaining against $i$. If $|S_i| \ge t$, then $i$ is marked as \textit{disqualified} by all players.
%                  Otherwise, if $|S_i| > 0$, then $i$ computes a \textit{KZG batch proof} $\kzgbatchproof_i$ for all $\{s_{i,j}\}_{j\in S_i}$ and broadcasts $(S_i, \{s_{i,j}\}_{j\in S_i}, \kzgbatchproof_i)$.
%            \item If the batch proof $\kzgbatchproof_i$ does not verify, then $i$ is disqualified. 
%                  Otherwise, each $j\in S_i$ now has his correct share $s_{i,j}$.
%            \item Let $Q$ denote the set of players that were \textit{not} disqualified.
%                  The agreed-upon (unknown) secret key $s=\sum_{j \in Q} z_j$.
%                  Each $i$ 
%                sets $c = \prod_{j\in Q} c_j$,
%                sets the \textit{public key} $g^s = \prod_{j\in Q} g^{z_j}$,
%                sets his share $s_i = \sum_{j\in Q} s_{j,i}$,
%                and sets his KZG proof $\pi_i = \prod_{j\in Q} \pi_{j,i}$.
%%            \begin{itemize}
%%                \item $P_i$ computes his \textit{verification key} (VK) $g^{s_i}$ and tweaks the proof $\pi_i$ to vouch for his $g^{s_i}$ rather than $s_i$.
%%                This allows $P_i$ to verifiably announce its VK to all other $P_j$'s
%%            \end{itemize}
%        \end{enumerate}

        \underline{Complaint round:}
        \begin{enumerate}
            \item Let $S_i$ be the set of players complaining against $i$. If $|S_i| \ge t$, then $i$ is marked as \textit{disqualified} by all honest players.
                  Otherwise, $i$ broadcasts $\{s_{i,j},\pi_{i,j}\}_{j\in S_i}$.
            % In some sense, the round has finished after the broadcast, and the verification work done below
            % should be considered as part of a new round. But since this verification is not followed by another broadcast, it is not considered as a new round.
            \item If any one proof does not verify (or $i$ did not broadcast), then $i$ is disqualified. 
                  Otherwise, each $j\in S_i$ now has his correct share $s_{i,j}$.
            \item Let $Q$ denote the set of players that were \textit{not} disqualified.
                  The agreed-upon (unknown) secret key $s=\sum_{j \in Q} z_j$.
                  Each $i$ 
                sets $c = \prod_{j\in Q} c_j$,
                sets the \textit{public key} $g^s = \prod_{j\in Q} g^{z_j}$,
                sets his share $s_i = \sum_{j\in Q} s_{j,i}$,
                and sets his KZG proof $\pi_i = \prod_{j\in Q} \pi_{j,i}$.
        \end{enumerate}
        
    %\textit{Postconditions:}
    %All honest participants agree on the public key $g^s$ and have a share $s_i$ of $s$ such that any subgroup of size $t$ participants can reconstruct $s$.
%\end{algorithm*}
\end{algorithm}
}

\ejfdkgAlgorithm

\subsection{Distributed Key Generation (DKG)}
\label{s:prelim:dkg}

TSS protocols pose a key generation problem: if one party $P$ splits $s$ to the $n$ signers (via SSS), $P$ would know $s$ and could sign on behalf of the group.
This would make the TSS insecure and thus motivates \textit{distributed key generation} (DKG) protocols~\cite{dkg,Pedersen1991AThreshold}.
A $(t,n)$ DKG protocol for discrete log cryptosystems allows $n$ players to jointly generate a secret key $s\in_R \Fp$ with public key $g^s\in \Group$ such that \textit{only} subsets of size $\ge t$ can reconstruct $s$.

Unlike VSS protocols, where the dealer knows $s$ (see \cref{s:prelim:vss}), DKG protocols guarantee nobody learns $s$ during the execution of the protocol.
Typically, DKG protocols achieve this by having each player $i$ secret-share its own secret $z_i$ and setting the \textit{final secret} $s$ to be $\sum_i z_i$.
Similar to VSS, DKG protocols must offer two security properties against any adversary who compromises $<t$ players: \textit{secrecy} and \textit{correctness}.
Informally, secrecy guarantees that no adversary can learn any information about $s$ beyond what is leaked by $g^s$.
Correctness guarantees that all honest players agree on $g^s$ and any subset with $\ge t$ honest players can reconstruct $s$.

\multipointEvalFigure

\subsubsection{Kate's \ejfdkg}
In this paper, we focus on improving \ejfdkg which, at a high-level, consists of $n$ parallel executions of \evss.
We describe only its sharing phase in \cref{alg:dkg}, since it has the same reconstruction phase as \evss.
Note that \ejfdkg makes use of \textit{non-interactive zero-knowledge proofs of knowledge (NIZKPoKs)}~\cite{CS97}.
Although \ejfdkg is \textit{biasable} and produces an $s$ that is not guaranteed to be uniform, computing discrete logs on $g^s$ is still hard~\cite{Kate2010Distributed,Gennaro2003SecureApplications}.
Also, debiasing DKG protocols is possible~\cite{dkg,NBB16,SJSW19}.

\subsection{Polynomial Multipoint Evaluation}
\label{s:prelim:multipoint-eval}

We build upon \textit{polynomial multipoint evaluation} techniques~\cite{moderncomputeralgebra-ch10}.
Given a degree $t$ polynomial $\phi$, naively evaluating it at $n > t$ points $x_1, \dots, x_n$ requires $\Theta(nt)$ time.
This is fast when $t$ is very small relative to $n$ but can be slow when $t \approx n$, as is the case in many instantiations of threshold cryptosystems.
Fortunately, a multipoint evaluation reduces this time to $O(n\log^2{n})$ using a divide and conquer approach.
Specifically, one first computes $\phi_L(x)=\phi(x) \bmod (x-x_1)(x-x_2)\cdots(x-x_{n/2})$ and then $\phi_R(x)=\phi(x) \bmod (x-x_{n/2+1})(x-x_{n/2+2})\cdots(x-x_n)$
Then, one simply recurses on the two \textit{half-sized} subproblems: evaluating $\phi_L(x)$ at $x_1, x_2, \dots, x_{n/2}$ and $\phi_R(x)$ at $x_{n/2+1}, x_{n/2+2},\dots x_n$.
Ultimately, the leaves of this recursive computation store $\phi(x) \bmod (x-x_i)$, which is exactly $\phi(i)$ by the polynomial remainder theorem (see \cref{f:multipoint-eval}).

For example, consider the multipoint evaluation of $\phi$ at $\{1,2,\dots,8\}$, which we depict in \cref{f:multipoint-eval}.
We start at the root node $\varepsilon$.
Here, we divide $\phi$ by the \textit{accumulator polynomial} $(x-1)(x-2)\dots(x-8)$ obtaining a \textit{quotient polynomial} $q_{1,8}$ and \textit{remainder polynomial} $r_{1,8}$.
Then, its left and right children divide $r_{1,8}$ by the left and right ``half'' of $(x-1)(x-2)\dots(x-8)$, respectively.
This proceeds recursively: each node $w$ divides $r_\parent(w)$ by its accumulator $a_w$, obtaining a quotient $q_w$ and remainder $r_w$ such that $r_{\parent(w)} = q_w a_w + r_w$.
% Note: Technically, if n > t, the accumulators close to the top of the tree will not be necessary, so some time could be saved there.
Note that all accumulator polynomials $a_w$ can be computed in $O(n\log^2{n})$ time by starting with the $(x-i)$ monomials as leaves of a binary tree and ``multiplying up the tree.''
Since division by a degree-bound $n$ accumulator takes $O(n\log{n})$ time, the total time is $T(n)=2T(n/2)+O(n\log{n}) = O(n\log^2{n})$~\cite{moderncomputeralgebra-ch10}.
%At each level $i$ in this tree ($i=0$ is the root), $2^i$ divisions are performed between polynomials of degree $t/2^i$ and $n/2^i$ in time $O\left(n/2^i \log{\left(n/2^i\right)}\right)$.
